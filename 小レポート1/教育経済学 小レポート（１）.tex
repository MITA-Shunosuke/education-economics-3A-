\documentclass[a4paper,10.5pt]{jarticle}

\usepackage[truedimen,top=25truemm,bottom=30truemm,hmargin=25truemm]{geometry}
\usepackage{calc}
\usepackage[dvipdfmx]{graphicx}
\usepackage{pxrubrica} %ルビをふる
\usepackage[dvipdfmx]{hyperref}
\usepackage{pxjahyper}
\newcommand{\ctext}[1]{\raise0.2ex\hbox{\textcircled{\scriptsize{#1}}}}

\begin{document}

%
%	表紙を必要としないもの用のTeX雛形(作成:@oz4point5)
%	おおむね社会学会の論文用の書式に従っている
%
%	文字組他に関しては以下のウェブページを参考にした
%	https://texwiki.texjp.org/?geometry
%

\makeatletter
\newcount\@chars\newcount\@lines
\@chars=40                      % 1行の文字数
\@lines=40                      % 1ページの行数
\newdimen\@kanjiskip
\@kanjiskip=\dimexpr(\textwidth-1zw*\@chars)/\numexpr\@chars-1
\newdimen\@@kanjiskip
\@@kanjiskip=\dimexpr\@kanjiskip/10
\setlength{\@tempdima}{1pt*\ratio{\dimexpr\textheight/\@lines}{\baselineskip}}
\renewcommand{\baselinestretch}{\strip@pt\@tempdima}\selectfont
\kanjiskip=\@kanjiskip plus \@@kanjiskip minus \@@kanjiskip
\parindent=\dimexpr 1zw+2truept
\parindent=\dimexpr\parindent+\@kanjiskip
\makeatother

%
%	タイトルなど
%

\title{教育経済学 小レポート(1)}
\author{三田 周之介(09-221211)\\東京大学 教育学部 比較教育社会学コース 3年}
\date{}
\maketitle

%
%	以下本文
\paragraph{問1}
\subparagraph{\ctext{1}}
60歳まで働き続けるというライフコースを仮定する場合の私的内部収益率

8.61\%

\subparagraph{\ctext{2}}
{\small 日本社会の女性にとって典型的だと思われるライフコースを仮定する場合の私的内部収益率}
\vskip\baselineskip
ここでは「29歳まで就労、30代は非就労、40歳から59歳まで就労」というパターンを考えてみる. この場合, \ctext{1}で用いたデータにおける30代の私的便益は0になる. それを用いて計算すると, 私的内部収益率は6.13\%となる. 

\subparagraph{\ctext{3}}
「限界的な選択に直面している女性(18歳)にとって大学進学が経済合理的な進路である」と言えるかどうか
\vskip\baselineskip
単純な私的収益率のみを考えると, 手元に大学進学を乗り切ることができる資本がある場合, 大学進学が経済合理的な進路であるといえる. 

\end{document}